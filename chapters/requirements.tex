\chapter{Requirements}

\label{chap:requiremnts}

We now define the requirements for the recommender system. Writing requirements can help to define, document and understand the software being produced \cite{sommerville2011software}. \cite{sommerville2011software} defines different types of requirements: user requirements and system requirements. User requirements are the requirements that are identified during the design phase and with people who may not have a technical understanding of the software to be developed. So user requirements are more focused on the goal of the software. We don't explicitly define user requirements because we have already given a motivation and a goal for the software.

\section{Functional Requirements}

\label{sec:func_requiremnts}

Below we define the functional requirements. This means that we define precise requirements for the system to meet, such as what the desired outcome is for a particular user input \cite{sommerville2011software}. Since we are developing an API, user input would be a request to the web server.

%\begin{table}[h]
%\begin{center}
\begin{longtable}{|p{1.2cm}||p{3.2cm}|p{9cm}|}
    \hline
	  FR & Name & Description \\
	\hline
	FR 1 & Application    & The application should be a REST Api written in Kotlin, accessible through the web \\
    \hline
	FR 2 & Authentication & Each Request to the API should be authenticated with a JSON web token \\
	\hline
    FR 3 & User Queries & The application should provide a route to any information about a user that the application has stored, such as the userid or username \\
	\hline
    FR 4 & User Mutation & The application should provide a Route to modify and add users \\
	\hline
    FR 5 & Item Queries & The application should provide a route for all information about an item that the application has stored\\
	\hline
    FR 6 & Item Mutations & The application should provide a Route to modify and add items \\
	\hline
    FR 7 & Rating Queries & The application should provide a route to get all the ratings of all users at once, or a specific user or item\\
	\hline
    FR 8 & User Similarity & The application should provide a route to get all similarities of a specific user with an http parameter to select a similarity algorithm and return a number \\
	\hline
    FR 9 & Item Similarity & The application should provide a route to get all similarities of a specific item with an http parameter to select a similarity algorithm and return a number  \\
	\hline
    FR 10 & User-based Recommendation & The application should provide a way to get an estimated rating for any item a particular user hasn't rated, based on other users' ratings \\
	\hline
    FR 11 & Item-based Recommendation & The application should provide a way to get an estimated rating for any item that a particular user hasn't rated based on item ratings \\
	\hline
    FR 12 & Recommendation Algorithms & The application should provide the Euclidean, Cosine, Adjusted Cosine and Pearson algorithms for possible similarity measures. \\
	\hline
    FR 13 & Data Saving & The application should persistently store ratings, items and users in a MySQL database \\
	\hline
    FR 14 & Implementation Technologies & The application should be implemented with Kotlin and the Ktor server framework \\
	\hline
    FR 15 & Data refreshment & The estimated ratings and similarities should be updated on request or when a rating is added \\
	\hline
    FR 16 & Serialization of Input and Output & The application should serialise and deserialise data into JSON for communication between clients\\
	\hline

  \caption{Functional Requirements}
  \label{tab:fas}
\end{longtable}

\section{Non-Functional Requirements}

\label{sec:non_func_requiremnts}

We now define the non-functional requirements, which are requirements that affect the system as a whole, or requirements that don't relate to a specific function of the software \cite{sommerville2011software}.

\begin{longtable}{|p{1.2cm}||p{3.2cm}|p{9cm}|}
    \hline
    NFR 1 & Performance & The software shouldn't take more then 1gb of RAM and should minimize network traffic.\\
    \hline
    NFR 2 & Testing & The software should test all similarity and recommendation algorithms in a meaningful way, including unexpected input data.\\
    \hline
    NFR 3 & Usage & The API should be easy for a potential developer to use through intuitive routes, query parameters and API documentation.\\
    \hline
    NFR 4 & Extensions & The software should be extensible with more similarity measures.\\
    \hline
  \caption{Non-functional Requirements}
  \label{tab:nfas}
\end{longtable}